


\chapter{Eager vs Lazy Loading}

Two distinctive strategies for loading related data in an object or entity. The terms are commonly used in the context of Object Relational Mapping(ORM) frameworks like Entity Framework.

\section{Eager Loading}
\begin{itemize}
\item The related data will be loaded immediately as part of the initial database query.
\item Useful when you know you're going to need the data right away, thus avoiding the need to make a subsequent database call.
\item Consumes more memory initially.
\end{itemize}

\begin{lstlisting}
// Eager Loading with Entity Framework
var students = context.Students
                      .Include(s => s.Courses) // Loads related Courses as well
                      .ToList();
\end{lstlisting}

\section{Lazy Loading}
\begin{itemize}
\item The related data will be loaded only when it's requested
\item Separate queries will be used to load the data.
\item Lower memory usage upfront.
\item Configured by default in EF6, however not configured in EFCore.
\item in EF Core we need to 
\begin{lstlisting}
dotnet add package Microsoft.EntityFrameworkCore.Proxies
\end{lstlisting}
and enable in code 
\begin{lstlisting}
public class MyDbContext : DbContext
{
    protected override void OnConfiguring(DbContextOptionsBuilder optionsBuilder)
    {
        optionsBuilder
            .UseSqlServer("YourConnectionString")
            .UseLazyLoadingProxies();  // Enable lazy loading proxies
    }
}
\end{lstlisting}
\end{itemize}

The lazily loaded entity can then be accessed when needed like this: 

\begin{lstlisting}
// Lazy Loading with Entity Framework (when enabled)
var student = context.Students.First(); // Only Student is loaded

var courses = student.Courses; // Courses are loaded when accessed
\end{lstlisting}

