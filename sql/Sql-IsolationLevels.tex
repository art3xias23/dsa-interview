\chapter*{Sql Indexes} % or \section*{} depending on your document class

Isolation levels represent the degree to which a transaction must be isolated from the data modifications made by other concurrent transactions in a database.

\section*{Read Uncommitted (Lowest Isolation Level} % Section for Header information
\begin{itemize}
    \item A transaction might read data that has been modified by another transaction, but not yet committed.
    \item Transaction 1 updates a balance from 100 to 150, but does not commit. Transaction 2 reads the 150 even though the value has not been committed yet.
    \item Reading uncommited data is knows as a \textbf{Dirty Read}
\end{itemize}

\section*{Read Committed}
\begin{itemize}
\item A transaction can only read committed data. That would avoid dirty reads, but could cause \textbf{Non-Repeatable Reads}.

\item Non-Repeatable Reads: The same data can be read differently in the same transaction. T1 starts reading a balance of 100, T2 updates it to 150. T1 finishes by reading the balance as 150. The value of the balance changed within the transaction.
\end{itemize}

\section*{Repeatable Read}
\begin{itemize}
\item This isolation level guarantees that the same value would be read within the transaction. T1 starts reading a balance of 100, T2 updates it to 150. This time T1 finishes by reading the balance as 100. Even though the value of balance changed the isolation level guarantees we will get Repeatable Reads. However, this isolation levels allows for \textbf{Phantom Reads}
\item Phantom reads: A transaction reads data that is no longer the latest committed version of that data
\end{itemize}

\section*{Serializable (Highest Isolation Level}
\begin{itemize}
\item Highest isolation level. Transactions are executed in a way that makes them appear as if they are running one after the other (serially).
\item While T1 is running, T2 will be blocked. That ensures T1 will not have Dirty Reads, Non-Repeatable Reads or Phantom Reads.
\end{itemize}


