\page{Hashes}

\section{Uses}

\begin{itemize}
    \item File Integrity
    \begin{itemize}
        \item Verify that a file has not been tempered with during a download
        \item Downloading software from a website often includes a checksum (e.g., SHA-256) that can be compared with the hash of the downloaded file to ensure it wasn’t corrupted or tampered with.
    \end{itemize}

    \item Password Storage
    \begin{itemize}
        \item Verify that a password is correct
        \item Instead of storing plain-text passwords, systems store the hashed value of the password. When a user logs in, the entered password is hashed, and the result is compared to the stored hash. 
    \end{itemize}

    \item Blockchain
    \begin{itemize}
        \item Verify that a transaction in a block has not been tapered with
        \item Hashes play a critical role in blockchain technologies by ensuring the integrity and immutability of transactions. In blockchain, every block contains the hash of the previous block, creating a chain. If any block is altered, the hash changes, breaking the chain and signaling tampering.
    \end{itemize}

    \item Version Control
    \begin{itemize}
        \item Verify that a commit has not been tampered with.
        \item Hashes are a picture of the state of the repository. It's objects, their metadata, even the commit message. As soon as one of them changes there will be a completely different commit hash.
        \item Can check repo integrity by running 

        \begin{lstlisting}
            git fsck
        \end{lstlisting}
    \end{itemize}
\end{itemize}