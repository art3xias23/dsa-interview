\chapter{Array}

\section{Definition}
\begin{itemize}
	\item Like arrays it's used to represent sequential data.
	\item Each element contains an address of the next element.
	\item A collection of nodes where each node contains data and a reference to the next node in the sequence.
\end{itemize}

\section{Advantages}
\begin{itemize}
	\item Deletion and Insertion of a node given it's location is constant O(1), where in arrays the items would have to be shifted.
\end{itemize}

\section{Disadvantages}
\begin{itemize}
	\item Accessing a node is linear time O(n).
	\item Accessing can only occur by traversing from the head. Access by index (as with arrays) not possible.
\end{itemize}

\section{Types}
\begin{itemize}
	\item Singly Linked List
			\begin{itemize}
				\item Each node points to the next node and the last node points to null.
			\end{itemize}
	\item Doubly Linked List
			\begin{itemize}
				\item Each node has two pointers. Next and Prev.
				\item The first node's Prev and the last node's Next pointers point to null.
			\end{itemize}
\end{itemize}

\section{Corner Cases}
\begin{itemize}
	\item Empty Sequence
	\item Sequence with 1 or 2 elements
	\item Sequence with repeated elements
	\item Duplicated values in the sequence
\end{itemize}

\begin{center}
\begin{tabular}{||c c c||} 
 \hline
 Operation & Complexity & Note\\ [0.5ex] 
 \hline\hline
 Access & O(n) \\ 
 \hline
 Search & O(n) \\
 \hline
 Insert & O(1) & Assumes you have traversed to the insertion position \\ 
 \hline
 Remove & O(1) & Assumes you have traversed to the node to be removed \\
 \hline

\end{tabular}
\end{center}

\section{Common Routines to Remember}
\begin{itemize}
	\item Counting the number of nodes in the linked list
	\item Reversing a linked list in-place
	\item Finding the middle node of the linked list using two pointers (fast/slow)
	\item Duplicated values in the sequence
\end{itemize}


\section{Corner Cases}
\begin{itemize}
	\item Empty linked list (head is null)
	\item Single node
	\item Two nodes
	\item Linked list has cycles. Tip: Clarify beforehand with the interviewer whether there can be a cycle in the list. Usually the answer is no and you don't have to handle it in the code
\end{itemize}