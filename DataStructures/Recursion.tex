\chapter{Recursion}
\section{Definition}
Recursion is a method of solving a computational problem where the solution depends on solutions to smaller instances of the same problem.
\section{Basics}
	\begin{itemize}
		\item Steps of a recursive function
			\begin{itemize}
			\item Base Case (i.e. when to stop)
			\item Work toward Base Case
			\item Recursive Call
			\end{itemize}
		\begin{lstlisting}
def fib(n):
	if n <= 1:
    return n
  return fib(n - 1) + fib(n - 2)
	\end{lstlisting}
	\end{itemize}
	
	\section{Corner Cases}
\begin{itemize}
	\item n = 0
	\item n = 1
\end{itemize}

		
		\section{Techniques}
		\begin{itemize}
			\item \textbf{Memoization}
			Memoization is an optimization technique which stores the results of expensive function calls and reuses them when the same inputs occur again.
		\begin{itemize}
			\item If you see a top or lowest k being mentioned in the question, it is usually a signal that a heap can be used to solve the problem, such as in \textbf{Top K Frequent Elements.}
			\item Stack with one item
			\item Stack with two items
		\end{itemize}
		\end{itemize}

